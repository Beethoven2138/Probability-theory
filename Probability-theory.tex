\documentclass{article}
\usepackage{graphicx} % Required for inserting images
\usepackage[english]{babel}
\usepackage{amssymb}
\usepackage{amsthm}
\usepackage{enumitem} 
\usepackage{amsmath}
\usepackage{amsfonts}
\usepackage{tikz}
\usetikzlibrary{matrix}
\usepackage[english]{babel}
\usepackage{mathtools}
\usepackage[a4paper, total={6in, 8in}]{geometry}
\usepackage{cite}


\newtheorem{theorem}{Theorem}[section]
\newtheorem{definition}[theorem]{Definition}
\newtheorem{lemma}[theorem]{Lemma}
\newtheorem{proposition}[theorem]{Proposition}
\newtheorem{corollary}[theorem]{Corollary}
\newtheorem{example}[theorem]{Example}
\newtheorem{remark}[theorem]{Remark}
\newtheorem{exercise}[theorem]{Exercise}


\title{Probability Theory}
\author{Saxon Supple}
\date{May 2025}

\begin{document}

\maketitle

\[\mathbb{P}(A|B)=\frac{\mathbb{P}(A\cap B)}{\mathbb{P}(B)}\] so \[\mathbb{P}(A|B)=\frac{1}{\mathbb{P}(B)}\int_A\mathbf{1}_Bd\mathbb{P}.\] Hence,\[\mathbb{E}[X|B]=\int Xd\mathbb{P}(\cdot |B)=\frac{1}{\mathbb{P}(B)}\int X\mathbf{1}_Bd\mathbb{P}=\frac{\mathbb{E}[X\mathbf{1}_B]}{\mathbb{P}(B)}.\]


Let $X$ be a non-negative integer-valued random variable. Then \[\mathbb{E}[X]=\sum_{i=1}^\infty\mathbb{P}(X\geq i).\]


\begin{exercise}
Let $(\Omega,\mathcal{F},\mathbb{P})$ be a probability space, and let $A_1,...,A_n$ be measurable sets. Prove by induction that\[\mathbb{P}(\bigcup_{i=1}^nA_i)\geq\sum_{i=1}^n\mathbb{P}(A_i)-\sum_{1\leq i < j\leq n}\mathbb{P}(A_i\cap A_j).\]
\end{exercise}
\begin{proof}
For $n=1$, \[\mathbb{P}\left(\bigcup_{i=1}^n A_i\right)=\mathbb{P}(A_1)\geq\sum_{i=1}^n\mathbb{P}(A_i)-\sum_{1\leq i < j\leq n}\mathbb{P}(A_i\cap A_j)=\mathbb{P}(A_1).\] Now assume true for $n=k$. Then for $n=k+1$ we have\begin{align*}
\mathbb{P}\left(\bigcup_{i=1}^{k+1}A_i\right)&=\mathbb{P}\left(A_{k+1}\cup\bigcup_{i=1}^{k}A_i\right)\\&=\mathbb{P}(A_{k+1})+\mathbb{P}\left(\bigcup_{i=1}^{k}A_i\right)-\mathbb{P}\left(\bigcup_{i=1}^{k}(A_i\cap A_{k+1})\right)\\&\geq\mathbb{P}(A_{k+1})+\sum_{i=1}^k\mathbb{P}(A_i)-\sum_{1\leq i<j\leq k}\mathbb{P}(A_i\cap A_j)-\mathbb{P}\left(\bigcup_{i=1}^{k}(A_i\cap A_{k+1})\right)\\&\geq\sum_{i=1}^{k+1}\mathbb{P}(A_i)-\sum_{1\leq i<j\leq k}\mathbb{P}(A_i\cap A_j)-\sum_{i=1}^k\mathbb{P}(A_i\cap A_{k+1})\\&=\sum_{i=1}^{k+1}\mathbb{P}(A_i)-\sum_{1\leq i<j\leq k+1}\mathbb{P}(A_i\cap A_j).
\end{align*}
Hence the result holds by induction.
\end{proof}

\begin{exercise}
Suppose $(\Omega,\mathcal{A},\mathbb{P})$ is a probability space and $X:\Omega\to[0,\infty)$ is a non-negative random variable.
\begin{enumerate}
    \item[(a)] Let $A_n=\{\omega\in\Omega:X(\omega)\leq n\}$. Prove that $\mathbb{E}[X\mathbf{1}_{A_n
    }]\to\mathbb{E}[X]$ as $n\to\infty$.
    \item[(b)] Prove that if $\mathbb{E}[X]<\infty$, then $\mathbb{E}[X\mathbf{1}_{A_n^c}]\to0$ as $n\to\infty$.
    \item[(c)] Suppose $\mathbb{E}[X]<\infty$. Prove that\[\lim_{\delta\downarrow0}\sup_{\{A\in\mathcal{F}:\mathbb{P}(A)<\delta\}}\mathbb{E}[X\mathbf{1}_A]=0.\]
\end{enumerate}
\end{exercise}
\begin{proof}
\begin{enumerate}
    \item[(a)] $0\leq X\mathbf{1}_{A_n}\uparrow X$ so apply the monotone convergence theorem.
    \item[(b)] $\mathbb{E}[X\mathbf{1}_{A_n}]+\mathbb{E}[X\mathbf{1}_{A_n^c}]=\mathbb{E}[X]\forall n$ so $\lim_{n\to\infty}\mathbb{E}[X\mathbf{1}_{A_n^c}]=\mathbb{E}[X]-\lim_{n\to\infty}\mathbb{E}[X\mathbf{1}_{A_n}]=0$.
    \item[(c)] Let $(\delta_n)_{n\in\mathbb{N}}$ be a sequence such that $\delta_n\downarrow0$. We want to show that given an $\epsilon > 0$ there exists an $N\in\mathbb{N}$ such that $\forall n>N$ we have\[\sup_{\{A\in\mathcal{F}:\mathbb{P}(A)<\delta_n\}}\mathbb{E}[X\mathbf{1}_A]<\epsilon.\] To do so, we instead show that there exists an $N\in\mathbb{N}$ such that $\forall n>N$ we have\[\mathbb{E}[X\mathbf{1}_A]<\frac{\epsilon}{2}\forall A\in\mathcal{F}:\mathbb{P}(A)<\delta_n,\] which would imply that \[\sup_{\{A\in\mathcal{F}:\mathbb{P}(A)<\delta_n\}}\mathbb{E}[X\mathbf{1}_A]\leq\frac{\epsilon}{2}<\epsilon.\] By part (b) there exists an $M\in\mathbb{N}$ such that \[\mathbb{E}[X\mathbf{1}_{\{ X>M\}}]<\frac{\epsilon}{4}.\] Let $N\in\mathbb{N}$ be such that $\delta_n<\frac{\epsilon}{4M}\forall n>N$. Then if $A\in\mathcal{F}$ is such that $\mathbb{P}(A)<\delta_n$, we have that\[X\mathbf{1}_A=X\mathbf{1}_{A\cap \{X>M\}}+X\mathbf{1}_{A\cap \{X\leq M\}}\]so\begin{align*}\mathbb{E}[X\mathbf{1}_A]&\leq \mathbb{E}[X\mathbf{1}_{\{X>M\}}]+\mathbb{E}[X\mathbf{1}_{A\cap\{X\leq M\}}]\\&<\frac{\epsilon}{4}+M\mathbb{P}(A)\\&<\frac{\epsilon}{4}+\frac{M\epsilon}{4M}=\frac{\epsilon}{2},\end{align*} as required.
\end{enumerate}
\end{proof}

\begin{exercise}
Let $\Omega=\{1,2,3,4,5,6\}$ with $\mathcal{A}=\mathcal{P}(\Omega)$ and let $\mathcal{C}=\{A,B\}$ with $A=\{1,2,3,4\}$ and $B=\{3,4,5\}$. What are the sets of $\sigma(\mathcal{C})$?
\end{exercise}
\begin{proof}
\[A\cap B=\{3,4\},A^c\cap B=\{5\},A\cap B^c=\{1,2\},A^c\cap B^c=\{6\}.\]
Hence, if we label the sets as \[X_1=\{3,4\},X_2=\{5\},X_3=\{1,2\},X_4=\{6\}\] then\[\sigma(\mathcal{C})=\{\bigcup_{i\in J}X_i:J\subseteq\{1,2,3,4\}\}.\]
\end{proof}

\begin{exercise}
Let $\Omega=\{1,2,3,4\}$ and let $\mathcal{A}$ be the $\sigma$-algebra $\sigma(\{\{1\},\{2\},\{3,4\}\})$ on $\Omega$.
\begin{enumerate}
    \item[(a)] Give an example of a subset of $\Omega$ that is not in $\mathcal{A}$.
    \item[(b)] Give an example of a function $f:\Omega\to\mathbb{R}$ that is not measurable.
\end{enumerate}
\end{exercise}
\begin{proof}
\begin{enumerate}
    \item[(a)] $\{1,4\}$.
    \item[(b)] $\mathbf{1}_{\{1,4\}}$.
\end{enumerate}
\end{proof}

\begin{exercise}
Let $(\Omega,\mathcal{A},\mathbb{P})$ be a probability space. Suppose that $\mathcal{C}$ is a finite partition of $\Omega$ with $\mathcal{C}\subset\mathcal{A}$, i.e. suppose that $\mathcal{C}=\{A_1,...,A_n\}$ with the sets $A_i$ partitioning $\Omega$, and with each $A_i$ in $\mathcal{A}$. Likewise, suppose that $\mathcal{B}=\{B_1,...,B_m\}$ is a finite partition of $\Omega$ with $\mathcal{B}\subset\mathcal{A}$.
\begin{enumerate}
    \item[(a)] Suppose that $\mathbb{P}(A\cap B)=\mathbb{P}(A)\mathbb{P}(B)$ for all events $A\in\mathcal{C}$ and $B\in\mathcal{B}$. Prove that $\sigma(\mathcal{C})$ and $\sigma(\mathcal{B})$ are independent $\sigma$-algebras.
    \item[(b)] Does the conclusion still hold if $\mathcal{C}=\{A_1,A_2,...\}$ and $\mathcal{B}=\{B_1,B_2,...\}$ are countably infinite partitions of $\Omega$ but otherwise things are as described above? Explain briefly.
\end{enumerate}
\end{exercise}
\begin{proof}
\begin{enumerate}
    \item[(a)] Let $A\in\sigma(\mathcal{C})$ and $B\in\sigma(\mathcal{B})$. Then $A=\bigcup_{i\in I}A_i$ and $B=\bigcup_{j\in J}B_j$ for some \[I\subseteq\{1,...,n\},J\subseteq\{1,...,m\}.\] Hence \begin{align*}\mathbb{P}(A\cap B)&=\mathbb{P}\left(\bigcup_{i\in I}A_i\cap\bigcup_{j\in J}B_j\right)\\&=\mathbb{P}\left(\bigcup_{i\in I}\left(A_i\cap \bigcup_{j\in J}B_j\right)\right)\\&=\sum_{i\in I}\mathbb{P}\left(A_i\cap\bigcup_{j\in J}B_j\right)\\&=\sum_{i\in I}\mathbb{P}\left(\bigcup_{j\in J}(A_i\cap B_j)\right)\\&=\sum_{i\in I}\sum_{j\in J}\mathbb{P}(A_i\cap B_j)\\&=\sum_{i\in I}\sum_{j\in J}\mathbb{P}(A_i)\mathbb{P}(B_j)\\&=\left(\sum_{i\in I}\mathbb{P}(A_i)\right)\left(\sum_{j\in J}\mathbb{P}(B_j)\right)\\&=\mathbb{P}(A)\mathbb{P}(B).\end{align*}
    \item[(b)] Yes, since the argument still holds with countably infinite indexes.
\end{enumerate}
\end{proof}

\begin{exercise}
\begin{enumerate}
    \item[(a)] Prove Markov's inequality, which says that if $X$  is a non-negative random variable and $a>0$, then $\mathbb{P}(X\geq a)\leq a^{-1}\mathbb{E}[X]$.
    \item[(b)] Prove that, if $X$ is a non-negative random variable with $\mathbb{E}[X]=0$, then $\mathbb{P}(X=0)=1$.
    \item[(c)] Suppose $X$ is a random variable with $\mathbb{E}[X]=\mu$ and $\text{Var}(X)=0$. Prove that $\mathbb{P}(X=\mu)=1$.
\end{enumerate}
\end{exercise}
\begin{proof}
\begin{enumerate}
    \item[(a)] \begin{align*}
        \mathbb{P}(X\geq a)&=\mathbb{E}[\mathbf{1}_{X\geq a}]\\&\leq\mathbb{E}\left[\frac{X}{a}\mathbf{1}_{X\geq a}\right]\\&\leq a^{-1}\mathbb{E}[X].
    \end{align*}
    \item[(b)] Let $\epsilon > 0$. Then $\mathbb{P}(X\geq\epsilon)\leq\epsilon^{-1}\cdot\mathbb{E}[X]=0$. $\epsilon$ is arbitrary so $\mathbb{P}(X>0)=\mathbb{P}(X\neq 0)=0$. Hence $\mathbb{P}(X=0)=1$.
    \item[(c)] $\text{Var}(X)=\mathbb{E}[(X-\mu)^2]=0$ so $\mathbb{P}((X-\mu)^2=0)=\mathbb{P}(X=\mu)=1$.
\end{enumerate}
\end{proof}

\begin{exercise}
Let $X$ and $Y$ be random variables on the same probability space $(\Omega,\mathcal{F},\mathbb{P})$.
\begin{enumerate}
    \item[(a)] Let $\mathcal{C}$ be the class of events of the form $\{X<t\},t\in\mathbb{R}$. Show that $\sigma(\mathcal{C})=\sigma(X)$.
    \item[(b)] Show that $\mathcal{C}$ is a $\pi$-system.
    \item[(c)] Suppose that $\mathbb{P}(\{X<a\}\cap\{Y<b\})=\mathbb{P}(X<a)\mathbb{P}(Y<b)$ for all real numbers $a$ and $b$. Let $\mathcal{C}$ be as in (a), and $\mathcal{C}'=\{\{Y<t\}:t\in\mathbb{R}\}$.
    \begin{itemize}
        \item[1.] Fix $A\in\mathcal{C}$ and set $\mu_A(B)=\mathbb{P}(A\cap B)$ and $\mu_A'(B)=\mathbb{P}(A)\mathbb{P}(B)$ for all $B\in\sigma(\mathcal{C}')$. Show that $\mu_A=\mu_A'$ on $\sigma(\mathcal{C}')$.
        \item[2.] Fix $B\in\sigma(\mathcal{C}')$, and set $\nu_B(A)=\mathbb{P}(A\cap B)$ and $\nu_B'(A)=\mathbb{P}(A)\mathbb{P}(B)$ for all $A\in\sigma(\mathcal{C})$. Show that $\nu_B=\nu_B'$ on $\sigma(\mathcal{C})$.
        \item[3.] Show that $X$ and $Y$ are independent random variables.
    \end{itemize}
\end{enumerate}
\end{exercise}
\begin{proof}
\begin{enumerate}
    \item[(a)] $(-\infty,t)$ is a Borel set $\forall t$ so $\mathcal{C}\subseteq\sigma(X)$, and hence $\sigma(\mathcal{C})\subseteq\sigma(X)$. $\mathcal{B}=\sigma(\mathcal{I})$, where $\mathcal{I}:=\{(-\infty,t):t\in\mathbb{R}\}$. Let $\mathcal{M}:=\{A\in\mathcal{B}:X^{-1}(A)\in\sigma(\mathcal{C})\}$. Clearly $\mathcal{I}\subseteq\mathcal{M}$. Let $A_1,A_2,...\in\mathcal{M}$. Then $X^{-1}(\bigcup_{i=1}^\infty A_i)=\bigcup_{i=1}^\infty X^{-1}(A_i)\in\sigma(\mathcal{C})$, since $\sigma(\mathcal{C})$ is a $\sigma$-algebra. Hence $\bigcup_{i=1}^\infty A_i\in\mathcal{M}$. Furthermore, $X^{-1}(\emptyset)=\emptyset\in\sigma(\mathcal{C})$ so $\emptyset\in\mathcal{M}$. Finally, Let $A\in\mathcal{M}$. Then $X^{-1}(A^c)=X^{-1}(A)^c\in\sigma(\mathcal{C})$ so $A^c\in\mathcal{M}$. Hence $\mathcal{M}$ is a $\sigma$-algebra so contains $\sigma(\mathcal{I})=\mathcal{B}$; that is, $X^{-1}(A)\in\sigma(\mathcal{C})\forall A\in\mathcal{B}$, so $\sigma(X)\subseteq\sigma(\mathcal{C})$. Hence $\sigma(X)=\sigma(\mathcal{C})$.
    \item[(b)] $\{X<t\}\cap\{X<s\}=\{X<\min(t,s)\}\in\mathcal{C}$.
    \item[(c)]
    \begin{itemize}
        \item[1.] $\mu_A$ and $\mu'_A$ agree on $\mathcal{C}'$, and $\mathcal{C}'$ is a $\pi$-system, so $\mu_A$ and $\mu'_A$ agree on $\sigma(\mathcal{C}')$ by the uniqueness lemma.
        \item[2.] $\nu_B$ and $\nu'_B$ agree on $\mathcal{C}$ by the above, and $\mathcal{C}$ is a $\pi$-system, so $\nu_B$ and $\nu'_B$ agree on $\sigma(\mathcal{C})$ by the uniqueness lemma.
        \item[3.] We have shown that $\mathbb{P}(A\cap B)=\mathbb{P}(A)\mathbb{P}(B)\forall A\in\sigma(\mathcal{C}),B\in\sigma(\mathcal{C}')$, and hence \[\mathbb{P}(A\cap B)=\mathbb{P}(A)\mathbb{P}(B)\forall A\in\sigma(X),B\in\sigma(Y),\] since $\sigma(\mathcal{C})=\sigma(X),\sigma(\mathcal{C}')=\sigma(Y)$. Thus $X$ and $Y$ are independent.
    \end{itemize}
\end{enumerate}
\end{proof}

\begin{exercise}
Let $\mathcal{F}$ be a sub-$\sigma$-algebra. Let $U$ and $V$ be two $\mathcal{F}$-measurable random variables.
\begin{enumerate}
    \item[(a)] Show that $U+V$ is $\mathcal{F}$-measurable.
    \item[(b)] Show that $UV$ is $\mathcal{F}$-measurable.
\end{enumerate}
\end{exercise}
\begin{proof}
\begin{enumerate}
    \item[(a)] $\{U+V<t\}=\bigcup_{s\in\mathbb{R}}\{U<s\}\cap\{V<t-s\}$. Furthermore, if $U(\omega)<s$ and $V(\omega)<t-s$ for some $s\in\mathbb{R}$, by the density of $\mathbb{Q}$ in $\mathbb{R}$ there will be some $q\in\mathbb{Q}$ such that $U(\omega)<q<s$ and hence $V(\omega)<t-s<t-q$, so $\omega\in\{U<q\}\cap\{V<t-q\}$. This is true for all $\omega\in\{U+V<t\}$ so $\{U+V<t\}\subseteq\bigcup_{q\in\mathbb{Q}}\{U<q\}\cap\{V<t-q\}$. The reverse inclusion clearly holds so \[\{U+V<t\}=\bigcup_{q\in\mathbb{Q}}\{U<q\}\cap\{V<t-q\}\in\mathcal{F}.\] Hence $U+V$ is $\mathcal{F}$-measurable.
    \item[(b)] Let $X$ be an $\mathcal{F}$-measurable random variable. Then $\{X^2<t\}=\{-\sqrt{t}<X<\sqrt{t}\}\in\mathcal{F}$ if $t\geq 0$, and $\{X^2<t\}=\emptyset\in\mathcal{F}$ otherwise. Hence $X^2$ is $\mathcal{F}$-measurable.
    Hence $UV=\frac{(U+V)^2-(U-V)^2}{4}$ is $\mathcal{F}$-measurable.
\end{enumerate}
\end{proof}
\begin{exercise}
Let $(X_i)_{i \geq 1}$ be a sequence of independent random variables such that $\mathbb{E}[|X_i|] < \infty$ for all $i \geq 1$. For all $n \geq 1$, set $Y_n = \prod_{i=1}^n X_i$.

\begin{enumerate}
    \item[(a)] Prove that, for all $n \geq 1$, $Y_n = \prod_{i=1}^n X_i$ is $\sigma(X_1, \ldots, X_n)$-measurable, and deduce that $\sigma(Y_n) \subset \sigma(X_1, \ldots, X_n)$.
    
    \item[(b)] Fix $n \geq 1$ and let
    \[
    \mathcal{C} = \left\{ \bigcap_{i=1}^n \{X_i \in B_i\} : B_1, \ldots, B_n \in \mathcal{B}(\mathbb{R}) \right\}.
    \]
    Show that $\mathcal{C}$ is a $\pi$-system.
    
    \item[(c)] Show that, for all $A \in \sigma(X_{n+1})$ and $C \in \mathcal{C}$, $\mathbb{P}(A \cap C) = \mathbb{P}(A)\mathbb{P}(C)$.
    
    \item[(d)] Conclude that, for all $A \in \sigma(X_{n+1})$ and $C \in \sigma(X_1, \ldots, X_n)$, $\mathbb{P}(A \cap C) = \mathbb{P}(A)\mathbb{P}(C)$.
    
    \item[(e)] Deduce that, for all $n \geq 1$, $Y_n$ and $X_{n+1}$ are independent.
    
    \item[(f)] By induction on $n$, prove that, for all $n \geq 1$,
    \[
    \mathbb{E} \left[ \prod_{i=1}^n X_i \right] = \prod_{i=1}^n \mathbb{E}[X_i]. \tag{2}
    \]
\end{enumerate}
\end{exercise}
\begin{proof}
\begin{enumerate}
    \item[(a)] $\sigma(X_i)\subseteq\sigma(X_1,...,X_n)\forall i$ so each $X_i$ is $\sigma(X_1,...,X_n)$-measurable. The product of measurable functions is measurable so $Y_n$ is $\sigma(X_1,...,X_n)$-measurable. It clearly follows that $\sigma(Y_n)\subseteq\sigma(X_1,...,X_n)$.
    \item[(b)] Let $A_1,...,A_n,B_1,...,B_n\in\mathcal{B}$. Then\[\bigcap_{i=1}^n\{X_i\in A_i\}\cap\bigcap_{i=1}^n\{X_i\in B_i\}=\bigcap_{i=1}^n\{X_i\in A_i\cap B_i\}\in\mathcal{C}.\]
    \item[(c)] Let $A=\{X_{n+1}\in B_{n+1}\}$ and let $C=\bigcap_{i=1}^n\{X_i\in B_i\}$ for some $B_1,...,B_{n+1}\in\mathcal{B}$. Then by independence we have \begin{align*}\mathbb{P}(A\cap C)&=\mathbb{P}(\bigcap_{i=1}^{n+1}\{X_i\in B_i\})\\&=\prod_{i=1}^{n+1}\mathbb{P}(X_i\in B_i)\\&=\mathbb{P}(A)\prod_{i=1}^n\mathbb{P}(X_i\in B_i)\\&=\mathbb{P}(A)\mathbb{P}(\bigcap_{i=1}^n\{X_i\in B_i\})\\&=\mathbb{P}(A)\mathbb{P}(C).\end{align*}
    \item[(d)] First note that $\sigma(X_i)\subseteq\mathcal{C}\forall i\in\{1,..,n\}$ so $\sigma(X_1)\cup...\cup\sigma(X_n)\subseteq\mathcal{C}$, and hence $\sigma(X_1,...,X_n)\subseteq\sigma(\mathcal{C})$, and furthermore that $\mathcal{C}\subseteq\sigma(X_1,...,X_n)$ so $\sigma(\mathcal{C})=\sigma(X_1,...,X_n)$. $\mathcal{C}$ is a $\pi$-system so apply the uniqueness lemma to the measures $\nu_A:\sigma(X_1,...,X_n)\to\mathbb{R}:C\mapsto\mathbb{P}(A\cap C)$ and $\nu_A':\sigma(X_1,...,X_n)\to\mathbb{R}:C\mapsto\mathbb{P}(A)\mathbb{P}(C)$ for every $A\in\sigma(X_{n+1})$.
    \item[(e)] Let $A\in\sigma(X_{n+1})$ and $C\in\sigma(Y_n)$. $\sigma(Y_n)\subseteq\sigma(X_1,...,X_n)$ so $\mathbb{P}(A\cap C)=\mathbb{P}(A)\mathbb{P}(C)$. Hence $Y_n$ and $X_{n+1}$ are independent.
    \item[(f)] The base case is clear. Assume true for $n=k$:\[\mathbb{E}\left[\prod_{i=1}^k X_i\right]=\prod_{i=1}^k\mathbb{E}[X_i].\] For $n=k+1$, we have that $X_{k+1}$ and $\prod_{i=1}^k X_i$ are independent so\[\mathbb{E}\left[\prod_{i=1}^{k+1}X_i\right]=\mathbb{E}\left[X_{k+1}\prod_{i=1}^kX_i\right]=\mathbb{E}[X_{k+1}]\mathbb{E}\left[\prod_{i=1}^kX_i\right]=\mathbb{E}[X_{k+1}]\prod_{i=1}^k\mathbb{E}[X_i]=\prod_{i=1}^{k+1}\mathbb{E}[X_i].\]
\end{enumerate}
\end{proof}
\begin{exercise}
Let $p \in (0,1)$. Let $X$ and $Y$ be two independent random variables on a probability space $(\Omega, \mathcal{A}, \mathbb{P})$, such that
\[
\mathbb{P}(X = 1) = \mathbb{P}(Y = 1) = p \quad \text{and} \quad \mathbb{P}(X = -1) = \mathbb{P}(Y = -1) = 1 - p.
\]

\begin{enumerate}
    \item[(a)] Let $A \in \mathcal{A}$ be an event. Give the formula for $\mathbb{E}[X \mid \sigma(A)]$ in terms of $\mathbb{E}[X \mid A]$ and $\mathbb{E}[X \mid A^c]$.
    
    \item[(b)] Show that $\mathbb{E}[X \mid X + Y = 0] = 0$.
    
    \item[(c)] Show that
    \[
    \mathbb{E}[X \mid X + Y \neq 0] = \frac{2p - 1}{p^2 + (1 - p)^2}.
    \]
    
    \item[(d)] Let $\mathcal{F}$ be the sub-$\sigma$-algebra generated by the event $\{X + Y = 0\}$. Calculate $\mathbb{E}[X \mid \mathcal{F}]$.
    
    \item[(e)] Calculate $\mathbb{E}[Y \mid \mathcal{F}]$. Are the random variables $\mathbb{E}[X \mid \mathcal{F}]$ and $\mathbb{E}[Y \mid \mathcal{F}]$ independent?\\
    \textit{(Hint: be aware that the answer might depend on the value of $p$.)}
\end{enumerate}
\end{exercise}
\begin{proof}
\begin{enumerate}
    \item[(a)] $\sigma(A)=\{A,A^c\}$. Hence \[\mathbb{E}[X|\sigma(A)]=\mathbb{E}[X|A]\mathbf{1}_A+\mathbb{E}[X|A^c]\mathbf{1}_{A^c}\] almost surely. To see this, first note that $\mathbb{E}[X|A]\mathbf{1}_A+\mathbb{E}[X|A^c]\mathbf{1}_{A^c}$ is $\sigma(A)$-measurable, and furthermore \begin{align*}\mathbb{E}[(\mathbb{E}[X|A]\mathbf{1}_A+\mathbb{E}[X|A^c]\mathbf{1}_{A^c})\mathbf{1}_A]&=\mathbb{E}[\mathbb{E}[X|A]\mathbf{1}_A]\\&=\mathbb{E}\left[\frac{\mathbb{E}[X\mathbf{1}_A]}{\mathbb{P}(A)}\mathbf{1}_A\right]\\&=\mathbb{E}[X\mathbf{1}_A],\end{align*} and similarly \[\mathbb{E}[(\mathbb{E}[X|A]\mathbf{1}_A+\mathbb{E}[X|A^c]\mathbf{1}_{A^c})\mathbf{1}_{A^c}]=\mathbb{E}[X\mathbf{1}_{A^c}].\]
    \item[(b)] \[\mathbb{E}[X|X+Y=0]=\frac{\mathbb{E}[X\mathbf{1}_{X+Y=0}]}{\mathbb{P}(X+Y=0)}.\]Since \[\{X+Y=0\}=\{X=1,Y=-1\}\sqcup\{X=-1,Y=1\}\] it follows that\begin{align*}
    \mathbb{E}[X\mathbf{1}_{X+Y=0}]&=\mathbb{P}(X=1,Y=-1)-\mathbb{P}(X=-1,Y=1)\\&=p(1-p)-(1-p)p=0.
    \end{align*}Hence\[\mathbb{E}[X|X+Y=0]=0.\]
    \item[(c)] \[\mathbb{E}[X|X+Y\neq0]=\frac{\mathbb{E}[X\mathbf{1}_{X+Y\neq0}]}{\mathbb{P}(X+Y\neq0)}.\] Since\[\{X+Y\neq0\}=\{X=1,Y=1\}\sqcup\{X=-1,Y=-1\},\] it follows that\begin{align*}
    \mathbb{E}[X\mathbf{1}_{X+Y\neq0}]&=\mathbb{P}(X=1,Y=1)-\mathbb{P}(X=-1,Y=-1)\\&=p^2-(1-p)^2\\&=p^2-1+2p-p^2=2p-1
    \end{align*}and\begin{align*}
    \mathbb{P}(X+Y\neq0)&=\mathbb{P}(X=1,Y=1)+\mathbb{P}(X=-1,Y=-1)\\&=p^2+(1-p)^2.
    \end{align*}
    Hence\[\mathbb{E}[X|X+Y\neq0]=\frac{2p-1}{p^2+(1-p)^2}.\]
    \item[(d)] \[\mathbb{E}[X|\mathcal{F}]=\frac{2p-1}{p^2+(1-p)^2}\mathbf{1}_{X+Y\neq0}.\]
    \item[(e)] \[\mathbb{E}[Y|\mathcal{F}]=\mathbb{E}[X|\mathcal{F}]\] by symmetry.
    First suppose that $\frac{2p-1}{p^2+(1-p)^2}\neq 0$.\[\mathbb{P}\left(\mathbb{E}[X|\mathcal{F}]=\frac{2p-1}{p^2+(1-p)^2},\mathbb{E}[Y|\mathcal{F}]=\frac{2p-1}{p^2+(1-p)^2}\right)=\mathbb{P}(X+Y\neq0)=p^2+(1-p)^2.\]and\[\mathbb{P}\left(\mathbb{E}[X|\mathcal{F}]=\frac{2p-1}{p^2+(1-p)^2}\right)\mathbb{P}\left(\mathbb{E}[Y|\mathcal{F}]=\frac{2p-1}{p^2+(1-p)^2}\right)=\mathbb{P}(X+Y\neq0)^2=(p^2+(1-p)^2)^2.\] Hence in order to have independence we need\begin{align*}&p^2+(1-p)^2=(p^2+(1-p)^2)^2\\\iff &p^2+(1-p)^2=1\\\iff &p^2-p=0\\\iff &p\in\{0,1\}.\end{align*}
    Now suppose $\frac{2p-1}{p^2+(1-p)^2}=0$. Then $\mathbb{E}[Y|\mathcal{F}]=\mathbb{E}[X|\mathcal{F}]=0$ almost surely so are independent. The only value of $p$ which gives this is $p=\frac{1}{2}$.
\end{enumerate}
\end{proof}

\begin{exercise}
Let $X$ and $Y$ be two random variables such that $\mathbb{E}[|X|] < \infty$ and $\mathbb{E}[|Y|] < \infty$. Assume that
\[
\mathbb{E}[X \mid Y] = Y \text{ a.s. and } \mathbb{E}[Y \mid X] = X \text{ a.s.}
\]

\begin{enumerate}
    \item[(a)] Prove that, for all $c \in \mathbb{Q}$, $\mathbb{E}[(X - Y) \mathbf{1}_{Y \leq c}] = 0$.
    
    \item[(b)] Deduce that, for all $c \in \mathbb{Q}$, $\mathbb{E}[(X - Y) \mathbf{1}_{\{X \leq c\} \cap \{Y \leq c\}}] \leq 0$.
    
    \item[(c)] Using the symmetry of $X$ and $Y$, deduce that, for all $c \in \mathbb{Q}$, 
    \[
    \mathbb{E}[(X - Y) \mathbf{1}_{\{X \leq c\} \cap \{Y \leq c\}}] = 0.
    \]
    
    \item[(d)] Deduce that, for all $c \in \mathbb{Q}$, $\mathbb{P}(X > c \text{ and } Y \leq c) = 0$.
    
    \item[(e)] Deduce that $X = Y$ almost surely.
\end{enumerate}
\end{exercise}
\begin{proof}
\begin{enumerate}
    \item[(a)] \[\mathbb{E}[(X-Y)\mathbf{1}_{Y\leq c}]=\mathbb{E}[X\mathbf{1}_{Y\leq c}]-\mathbb{E}[Y\mathbf{1}_{Y\leq c}]=\mathbb{E}[X\mathbf{1}_{Y\leq c}]-\mathbb{E}[\mathbb{E}[X|Y]\mathbf{1}_{Y\leq c}].\]Since \[\{Y\leq c\}\in\sigma(Y)\] it follows that \[\mathbb{E}[\mathbb{E}[X|Y]\mathbf{1}_{Y\leq c}]=\mathbb{E}[X\mathbf{1}_{Y\leq c}]\] so \[\mathbb{E}[(X-Y)\mathbf{1}_{Y\leq c}]=0.\]
    \item[(b)] \begin{align*}
        \mathbb{E}[(X-Y)\mathbf{1}_{\{X\leq c\}\cap\{Y\leq c\}}]&=\mathbb{E}[(X-Y)\mathbf{1}_{\{Y\leq c\}}]-\mathbb{E}[(X-Y)\mathbf{1}_{\{X> c\}\cap\{Y\leq c\}}]\\&=-\mathbb{E}[(X-Y)\mathbf{1}_{\{X> c\}\cap\{Y\leq c\}}]\leq 0.
    \end{align*}
    \item[(c)] It follows from symmetry that \[\mathbb{E}[(Y-X)\mathbf{1}_{\{X\leq c\}\cap\{Y\leq c\}}]\leq0\] so \[\mathbb{E}[(X-Y)\mathbf{1}_{\{X\leq c\}\cap\{Y\leq c\}}]\geq 0\], and hence \[\mathbb{E}[(X-Y)\mathbf{1}_{\{X\leq c\}\cap\{Y\leq c\}}]=0.\]
    \item[(d)] \[\mathbb{E}[(X-Y)\mathbf{1}_{\{X\leq c\}\cap\{Y\leq c\}}]+\mathbb{E}[(X-Y)\mathbf{1}_{\{X>c\}\cap\{Y\leq c\}}]=\mathbb{E}[(X-Y)\mathbf{1}_{\{Y\leq c\}}]=0\]so\[\mathbb{E}[(X-Y)\mathbf{1}_{\{X>c\}\cap\{Y\leq c\}}]=0.\]Since\[(X-Y)\mathbf{1}_{\{X>c\}\cap\{Y\leq c\}}\geq0,\] it follows that \[(X-Y)\mathbf{1}_{\{X>c\}\cap\{Y\leq c\}}=0\] almost surely. That is,\[\mathbb{P}((X-Y)\mathbf{1}_{\{X>c\}\cap\{Y\leq c\}}>0)=\mathbb{P}(X>c,Y\leq c)=0.\]
    \item[(e)] \[\{X\neq Y\}=\bigcup_{c\in \mathbb{Q}}\{X>c,Y\leq c\}\cup\bigcup_{c\in\mathbb{Q}}\{Y>c,X\leq c\}.\]Hence,\[\mathbb{P}(X\neq Y)=\sum_{c\in\mathbb{Q}}\mathbb{P}(X>c,Y\leq c)+\sum_{c\in\mathbb{Q}}\mathbb{P}(Y>c,X\leq c)=0\] so\[X=Y\] almost surely.
\end{enumerate}
\end{proof}

\begin{exercise}
Assume that we toss $n$ fair coins and $N$ of them come up as heads. Suppose we then roll $N$ fair dice. Let $T$ be the sum of the values obtained on the $N$ dice. Express $\mathbb{E}[T|N]$ as a function of $N$, and then express $\mathbb{E}[T]$ and $\mathbb{E}[NT]$, both as functions of $n$.
\end{exercise}
\begin{proof}
\[\sigma(N)=\{\bigcup_{i\in I}\{N=i\}:I\subseteq\{0,...,n\},\] with \[\{N=0\},...,\{N=n\}\] being a partition of $\Omega$, and hence\begin{align*}\mathbb{E}[T|N]&=\sum_{i=0}^n\mathbb{E}[T|N=i]\mathbf{1}_{N=i}\\&=\sum_{i=0}^n3.5i\mathbf{1}_{N=i}\\&=3.5N.\end{align*}

Next,\[\mathbb{E}[T]=\mathbb{E}[T|N]=3.5\mathbb{E}[N]=3.5\cdot\frac{1}{2}n=1.75n.\]

Finally, \[\mathbb{E}[NT|N]=N\mathbb{E}[T|N]=3.5N^2.\] Hence\begin{align*}\mathbb{E}[NT]&=\mathbb{E}[\mathbb{E}[NT|N]]\\&=\mathbb{E}[3.5N^2]\\&=3.5\sum_{i=0}^ni^2\mathbb{P}(N=i)\\&=3.5\sum_{i=0}^ni^2{{n}\choose{i}}\frac{1}{2}^i\frac{1}{2}^{n-i}\\&=\frac{3.5}{2^n}\sum_{i=0}^ni^2{{n}\choose{i}}.\end{align*}
\end{proof}

\begin{exercise}
The aim of this question is to prove that, if \(X\) and \(Y\) are independent random variables and that both \(X\) and \(Y\) are in \(L^1\) (i.e. \(\mathbb{E}[|X|] < \infty\) and \(\mathbb{E}[|Y|] < \infty\)), then
\[
XY \in L^1 \quad \text{and} \quad \mathbb{E}[XY] = \mathbb{E}[X] \mathbb{E}[Y]. \tag{1}
\]
In particular, if \(X\) and \(Y\) are independent random variables in \(L^2\), then
\[
\text{Cov}(X,Y) = 0 \quad \text{and} \quad \text{Var}(X + Y) = \text{Var}(X) + \text{Var}(Y).
\]

(Note that, without the independence assumption, in general \(X, Y \in L^1\) does \emph{not} imply \(XY \in L^1\).)

\begin{itemize}
    \item[(a)] Prove (1) under the assumption that \(X\) and \(Y\) are non-negative and bounded, i.e. there exists some \(K\) such that \(0 \leq X \leq K\) and \(0 \leq Y \leq K\) almost surely.
    \item[(b)] Prove (1) under the assumption that \(X\) and \(Y\) are non-negative. (Hint: set \(X_n = \min(X, n)\) and \(Y_n = \min(Y, n)\), for all \(n \geq 1\).)
    \item[(c)] Prove (1) in full generality.
\end{itemize}
\end{exercise}
\begin{proof}
\begin{enumerate}
    \item[(a)] \[\mathbb{E}[XY]\leq\mathbb{E}[KY]=K\mathbb{E}[Y]<\infty\] so \[XY\in L^1.\]
    Hence,\[\mathbb{E}[XY]=\mathbb{E}[\mathbb{E}[XY|X]]=\mathbb{E}[X\mathbb{E}[Y|X]]=\mathbb{E}[X\mathbb{E}[Y]]=\mathbb{E}[X]\mathbb{E}[Y].\]
    \item[(b)] $X_n$ is $\sigma(X)$-measurable $\forall n\in\mathbb{N}$, since\[X_n=n\mathbf{1}_{\{X>n\}}+X\mathbf{1}_{\{X\leq n\}}.\] Hence \[\sigma(X_n)\subseteq\sigma(X)\forall n\]. Similarly, \[\sigma(Y_n)\subseteq\sigma(Y)\forall n\] and hence $X_n$ and $Y_n$ are independent $\forall n\in\mathbb{N}$. Hence, by (a), \[\mathbb{E}[X_nY_n]=\mathbb{E}[X_n]\mathbb{E}[Y_n]\forall n\in\mathbb{N}.\]
    Furthermore,
    \[0\leq X_n\uparrow X\text{ and }0\leq Y_n\uparrow Y\]so by the monotone convergence theorem
    \[\mathbb{E}[XY]=\mathbb{E}[\lim_{n\to\infty}X_nY_n]=\lim_{n\to\infty}\mathbb{E}[X_nY_n]=\lim_{n\to\infty}\mathbb{E}[X_n]\mathbb{E}[Y_n]=\mathbb{E}[X]\mathbb{E}[Y].\]
    \item[(c)] We have \[X=X^+-X^-,Y=Y^+-Y^-\] so \begin{align*}\mathbb{E}[XY]&=\mathbb{E}[(X^+-X^-)(Y^+-Y^-)]\\&=\mathbb{E}[X^+Y^+-X^+Y^--X^-Y^++X^-Y^-]\\&=\mathbb{E}[X^+]\mathbb{E}[Y^+]-\mathbb{E}[X^+]\mathbb{E}[Y^-]-\mathbb{E}[X^-]\mathbb{E}[Y^+]+\mathbb{E}[X^-]\mathbb{E}[Y^-]\\&=(\mathbb{E}[X^+]-\mathbb{E}[X^-])(\mathbb{E}[Y^+]-\mathbb{E}[Y^-])\\&=\mathbb{E}[X]\mathbb{E}[Y].\end{align*}
\end{enumerate}
\end{proof}

\begin{exercise}
Let $(\Omega, \mathcal{A}, \mathbb{P})$ be a probability space. Let $\mathcal{F} \subset \mathcal{G}$ be two sub-$\sigma$-algebras of $\mathcal{A}$, and $X$ a random variable on $(\Omega, \mathcal{A}, \mathbb{P})$.

\begin{itemize}
  \item[(a)] Prove that $\mathbb{E}[X \mathbb{E}[X \mid \mathcal{F}]] = \mathbb{E}[\mathbb{E}[X \mid \mathcal{F}]^2]$.
  
  \item[(b)] Prove that $\mathbb{E}[\mathbb{E}[X \mid \mathcal{F}] \mathbb{E}[X \mid \mathcal{G}]] = \mathbb{E}[\mathbb{E}[X \mid \mathcal{F}]^2]$.
  
  \item[(c)] Deduce from (i) and (ii) that
  \[
  \mathbb{E}[(X - \mathbb{E}[X \mid \mathcal{G}])^2] + \mathbb{E}[(\mathbb{E}[X \mid \mathcal{G}] - \mathbb{E}[X \mid \mathcal{F}])^2] = \mathbb{E}[(X - \mathbb{E}[X \mid \mathcal{F}])^2].
  \]
  
  \item[(d)] Recall that $\operatorname{Var}(X) = \mathbb{E}[(X - \mathbb{E}[X])^2]$. Similarly, we define, for all sub-$\sigma$-algebras $\mathcal{F} \subset \mathcal{A}$,
  \[
  \operatorname{Var}(X \mid \mathcal{F}) := \mathbb{E}[(X - \mathbb{E}[X \mid \mathcal{F}])^2 \mid \mathcal{F}].
  \]
  Taking $\mathcal{F} = \{\emptyset, \Omega\}$ in (iii), show that, for any sub-$\sigma$-algebra $\mathcal{G} \subset \mathcal{A}$,
  \[
  \operatorname{Var}(X) = \mathbb{E}[\operatorname{Var}(X \mid \mathcal{G})] + \operatorname{Var}(\mathbb{E}[X \mid \mathcal{G}]).
  \tag{2}
  \]
  
  (Explain why $\mathcal{F} \subset \mathcal{G}$ for any sub-$\sigma$-algebra $\mathcal{G} \subset \mathcal{A}$.)
\end{itemize}
\end{exercise}
\begin{proof}
\begin{enumerate}
    \item[(a)] \begin{align*}
        \mathbb{E}[X\mathbb{E}[X|\mathcal{F}]]&=\mathbb{E}[\mathbb{E}[X\mathbb{E}[X|\mathcal{F}]|\mathcal{F}]]\\&=\mathbb{E}[\mathbb{E}[X|\mathcal{F}]\mathbb{E}[X|\mathcal{F}]]\\&=\mathbb{E}[\mathbb{E}[X|\mathcal{F}]^2].
    \end{align*}
    \item[(b)] \begin{align*}
        \mathbb{E}[\mathbb{E}[X|\mathcal{F}]\mathbb{E}[X|\mathcal{G}]]&=\mathbb{E}[\mathbb{E}[\mathbb{E}[X|\mathcal{F}]X|\mathcal{G}]]\\&=\mathbb{E}[\mathbb{E}[X|\mathcal{F}]X]\\&=\mathbb{E}[\mathbb{E}[X|\mathcal{F}]^2].
    \end{align*}
    \item[(c)] \begin{align*}
        \mathbb{E}[(X - \mathbb{E}[X \mid \mathcal{G}])^2] + \mathbb{E}[(\mathbb{E}[X \mid \mathcal{G}] - \mathbb{E}[X \mid \mathcal{F}])^2]&=\mathbb{E}[X^2-2X\mathbb{E}[X|\mathcal{G}]+2\mathbb{E}[X|\mathcal{G}]^2-2\mathbb{E}[X|\mathcal{G}]\mathbb{E}[X|\mathcal{F}]+\mathbb{E}[X|\mathcal{F}]^2]\\&=\mathbb{E}[X^2-2\mathbb{E}[X|\mathcal{G}]^2+2\mathbb{E}[X|\mathcal{G}]^2-2\mathbb{E}[X|\mathcal{F}]^2+\mathbb{E}[X|\mathcal{F}]^2]\\&=\mathbb{E}[X^2-\mathbb{E}[X|\mathcal{F}]^2]\\&=\mathbb{E}[X^2-2X\mathbb{E}[X|\mathcal{F}]^2+\mathbb{E}[X|\mathcal{F}]^2]\\&=\mathbb{E}[(X-\mathbb{E}[X|\mathcal{F}])^2].
    \end{align*}
    \item[(d)] If $\mathcal{F}=\{\emptyset,\Omega\}$ then $\mathbb{E}[X|\mathcal{F}]=\mathbb{E}[X]$ almost surely. Hence\[\text{Var}(X)=\mathbb{E}[(X-\mathbb{E}[X|\mathcal{F}])^2].\] Furthermore,\[\mathbb{E}[(X-\mathbb{E}[X|\mathcal{G}])^2]=\mathbb{E}[\mathbb{E}[(X-\mathbb{E}[X|\mathcal{G}])^2]|\mathcal{G}]=\mathbb{E}[\text{Var}(X|\mathcal{G})]\] and\begin{align*}\mathbb{E}[(\mathbb{E}[X \mid \mathcal{G}] - \mathbb{E}[X \mid \mathcal{F}])^2]&=\mathbb{E}[(\mathbb{E}[X \mid \mathcal{G}] - \mathbb{E}[X ])^2]\\&=\mathbb{E}[(\mathbb{E}[X \mid \mathcal{G}] - \mathbb{E}[\mathbb{E}[X|\mathcal{G}]])^2]\\&=\text{Var}(\mathbb{E}[X|\mathcal{G}]).\end{align*}
    Hence \[\operatorname{Var}(X) = \mathbb{E}[\operatorname{Var}(X \mid \mathcal{G})] + \operatorname{Var}(\mathbb{E}[X \mid \mathcal{G}]).\]
\end{enumerate}
\end{proof}

\begin{exercise}
\textbf{Martingales associated with the simple random walk.}  
Let $p \in (0,1)$ and $(S_n)_{n \geq 1}$ be the simple random walk with parameter $p$ starting at $0$, i.e.
\[
S_n = \sum_{i=1}^n X_i,
\]
where $(X_i)_{i \geq 1}$ is a sequence of independent random variables satisfying  
\[
\mathbb{P}(X_i = 1) = p \quad \text{and} \quad \mathbb{P}(X_i = -1) = 1 - p \quad (\forall i \geq 1).
\]
Let $(\mathcal{F}_n)_{n \geq 1}$ be the natural filtration of $(X_i)_{i \geq 1}$.  
For all $n \geq 0$, define
\[
M_n := S_n - n\alpha, \quad M'_n := \beta^{S_n}, \quad \text{and} \quad M''_n := S_n^2 - n.
\]
\begin{enumerate}
\item[(a)] Find $\alpha$ and $\beta$ such that $(M_n)_{n \geq 1}$ and $(M'_n)_{n \geq 1}$ are martingales with respect to $(\mathcal{F}_n)_{n \geq 1}$.

\item[(b)] Show that if $p = 1/2$, then $(M''_n)_{n \geq 1}$ is a martingale with respect to $(\mathcal{F}_n)_{n \geq 1}$.
\end{enumerate}
\end{exercise}
\begin{proof}
Given any $\alpha$, $M_n$ is $\mathcal{F}_n$-measurable for all $n$. Furthermore, \[\mathbb{E}[|M_n|]=\mathbb{E}[|S_n-n\alpha|]\leq\mathbb{E}[|S_n|]+n|\alpha|\leq n+n|\alpha|<\infty,\] so $M_n$ is integrable for every $n$ and $\alpha$. We now need\[\mathbb{E}[M_{n+1}|\mathcal{F}_n]=M_n\text{ almost surely.}\]\begin{align*}\mathbb{E}[M_{n+1}|\mathcal{F}_n]&=\mathbb{E}[S_{n+1}-(n+1)\alpha|\mathcal{F}_n]\\&=\mathbb{E}\left[X_{n+1}+S_n-(n+1)\alpha\mid\mathcal{F}_n\right]\\&=\mathbb{E}[X_{n+1}
|\mathcal{F}_n]+S_n-(n+1)\alpha\\&=\mathbb{E}[X_{n+1}]+S_n-(n+1)\alpha\\&=2p-1+S_n-(n+1)\alpha\\&=M_n+2p-1-\alpha\text{ almost surely}.\end{align*}Hence $M_n$ is a martingale when \[2p-1-\alpha=0\iff \alpha=2p-1.\]

\noindent$f:\mathbb{R}\to\mathbb{R}:x\mapsto\beta^x$ is continuous $\forall \beta$ so $M_n'$ is $\mathcal{F}_n$-measurable $\forall n,\beta$.\[\mathbb{E}[|\beta^{S_n}|]\leq\mathbb{E}[|\beta^n|]<\infty\] so $M_n'$ is integrable.\begin{align*}
\mathbb{E}[M_{n+1}'|\mathcal{F}_n]&=\mathbb{E}[\beta^{S_{n+1}}|\mathcal{F}_n]\\&=\mathbb{E}[\beta^{X_{n+1}}\beta^{S_n}|\mathcal{F}_n]\\&=\beta^{S_n}\mathbb{E}[\beta^{X_{n+1}}|\mathcal{F}_n]\\&=M_n'\mathbb{E}[\beta^{X_{n+1}}]\\&=M_n'(p\beta+(1-p)\beta^{-1})\\&=M_n'.
\end{align*}
Hence, we need\begin{align*}
&p\beta+(1-p)\beta^{-1}=1\\\iff &p\beta^2-\beta+(1-p)=0\\\iff&\beta=\frac{1\pm\sqrt{1-4p(1-p)}}{2p}\\\iff&\beta=\frac{1\pm\sqrt{4p^2-4p+1}}{2p}\\\iff&\beta=\frac{1\pm|2p-1|}{2p}.
\end{align*} Hence either \[\beta=\frac{1+2p-1}{2p}=1\] or \[\beta=\frac{1-2p+1}{2p}=\frac{1-p}{p}.\]
\item[(b)] $M_n''$ is integrable, since\[\mathbb{E}[|M_n''|]\leq\mathbb{E}[|S_n^2|]+n\leq\mathbb{E}[|n^2|]+n<\infty.\] $M_n''$ is also clearly $\mathcal{F}_n$-measurable. \begin{align*}
\mathbb{E}[M_{n+1}''|\mathcal{F}_n]&=\mathbb{E}[S_{n+1}^2-n-1|\mathcal{F}_n]\\&=\mathbb{E}[(S_n+X_{n+1})^2-n-1|\mathcal{F}_n]\\&=\mathbb{E}[S_n^2+2S_nX_{n+1}+X_{n+1}^2-n-1|\mathcal{F}_n]\\&=M_n''+\mathbb{E}[2S_nX_{n+1}+X_{n+1}^2-1|\mathcal{F}_n]\\&=M_n''+2S_n\mathbb{E}[X_{n+1}]+\mathbb{E}[X_{n+1}^2]-1\\&=M_n''+0+\frac{1}{2}\cdot 1+\frac{1}{2}\cdot(-1)^2-1\\&=M_n''.
\end{align*}
Hence $M_n''$ is a martingale.
\end{proof}

\begin{exercise}
\textbf{Martingales associated with a Galton-Watson branching process.} Assume that $(Z_n)_{n\geq0}$ is the Galton-Watson branching process of offspring distribution $(p_i)_{i\geq0}$. Assume that $\mu=\sum_{i=1}^\infty ip_i\in(0,\infty)$. Let $(\mathcal{F}_n)_{n\geq1}$ be the natural filtration of $(Z_n)_{n\geq0}$.
\begin{enumerate}
    \item[(a)] For all $n\geq0$, set $M_n=\frac{Z_n}{\mu^n}$. Show that $(M_n)_{n\geq0}$ is a martingale with respect to $(\mathcal{F}_n)_{n\geq0}$.
    \item[(b)] Assume that $\alpha>0$ is such that $\sum_{i=0}^\infty\alpha^ip_i=\alpha$. For all $n\geq1$, set $M_n'=\alpha^{Z_n}$. Show that $(M_n')_{n\geq1}$ is a martingale with respect to $(\mathcal{F}_n)_{n\geq1}$.
\end{enumerate}
\end{exercise}
\begin{proof}
\begin{enumerate}
    \item[(a)] We have that \[\mathbb{E}[Z_{n+1}|\mathcal{F}_n]=\mu Z_n\forall n\geq 0\] so \[\mathbb{E}\left[\frac{Z_{n+1}}{\mu^{n+1}}|\mathcal{F}_n\right]=\frac{\mu Z_n}{\mu^{n+1}}=\frac{Z_n}{\mu^n}\] and hence $\frac{Z_n}{\mu^n}$ is a martingale with respect to $(\mathcal{F}_n)_{n\geq0}$.
    \item[(b)] First note that \[\mathbb{E}[\alpha^{\zeta_{n,i}}]=\sum_{i=0}^\infty\alpha^ip_i=\alpha.\]$M_n'$ is $\mathcal{F}_n$-measurable, since $f:\mathbb{R}\to\mathbb{R}:x\mapsto\alpha^x$ is continuous.
    
    We now show that $M_n'$ is integrable $\forall n$. Clearly $M_1'=\alpha^{\zeta_{0,1}}$ is integrable. Now assume $M_n'$ is integrable. We then have
    \begin{align*}\mathbb{E}[|M_{n+1}'|]&=\mathbb{E}[\alpha^{Z_{n+1}}]\\&=\mathbb{E}[\alpha^{\sum_{i=1}^{Z_n}\zeta_{n,i}}]\\&=\mathbb{E}[\sum_{k=0}^\infty\mathbf{1}_{\{Z_n=k\}}\alpha^{\sum_{i=1}^k\zeta_{n,i}}]\\&=\sum_{k=0}^\infty\mathbb{E}[\mathbf{1}_{\{Z_n=k\}}\alpha^{\sum_{i=1}^k\zeta_{n,i}}].\end{align*} For each $k$, we have\[\alpha^{\sum_{i=1}^k\zeta_{n,i}}=\prod_{i=1}^k\alpha^{\zeta_{n,i}}\in L^1,\] since $\alpha^{\zeta_{n,i}}\in L^1\forall n,i$. Hence, by independence,\begin{align*}
    \sum_{k=0}^\infty\mathbb{E}[\mathbf{1}_{\{Z_n=k\}}\alpha^{\sum_{i=1}^k\zeta_{n,i}}]&=\sum_{k=0}^\infty\mathbb{E}[\mathbf{1}_{\{Z_n=k\}}]\mathbb{E}[\alpha^{\sum_{i=1}^k\zeta_{n,i}}]\\&=\sum_{k=0}^\infty\mathbb{P}(Z_n=k)\alpha^k\\&=\mathbb{E}[\alpha^{Z_n}]<\infty.
    \end{align*} Hence $M_n'$ is integrable $\forall n$ by induction.
    Finally,
    \begin{align*}\mathbb{E}[M_{n+1}'|\mathcal{F}_n]&=\mathbb{E}[\alpha^{Z_{n+1}}|\mathcal{F}_n]\\&=\mathbb{E}[\alpha^{\sum_{i\geq1}\zeta_{n,i}\mathbf{1}_{\{i\leq Z_n\}}}|\mathcal{F}_n]\\&=\mathbb{E}\left[\sum_{k=1}^\infty\mathbf{1}_{\{Z_n=k\}}\alpha^{\sum_{i=1}^k\zeta_{n,i}}|\mathcal{F}_n\right]\\&=\sum_{k=0}^\infty\mathbb{E}[\mathbf{1}_{\{Z_n=k\}}\alpha^{\sum_{i=1}^k\zeta_{n,i}}|\mathcal{F}_n]\\&=\sum_{k=0}^\infty\mathbf{1}_{\{Z_n=k\}}\mathbb{E}[\alpha^{\sum_{i=1}^k\zeta_{n,i}}|\mathcal{F}_n]\\&=\sum_{k=0}^\infty\mathbf{1}_{\{Z_n=k\}}\mathbb{E}[\prod_{i=1}^k\alpha^{\zeta_{n,i}}]\\&=\sum_{k=0}^\infty\mathbf{1}_{\{Z_n=k\}}\alpha^k\\&=\alpha^{Z_n}\\&=M_n'\text{ almost surely.}\end{align*}Hence $M_n'$ is a martingale with respect to $(\mathcal{F}_n)_{n\geq1}$.
\end{enumerate}
\end{proof}

\begin{exercise}[Generalised Pólya urn]
An urn initially contains $X_0 = x > 0$ grams of red powder and $Y_0 = y > 0$ grams of green powder.  Sample $(V_i)_{i\ge1}$, a sequence of non-negative i.i.d.\ random variables.  At each time step $n+1$ ($n\ge0$), we pick a speck of powder uniformly at random from the urn and then return it to the urn together with $V_{n+1}$ grams of powder of the same colour as that speck.

More formally, let $(U_i)_{i\ge1}$ be a sequence of i.i.d.\ $\mathrm{Unif}[0,1]$ random variables, independent of $(V_i)_{i\ge1}$.  Define $(X_n)_{n\ge0}$ and $(Y_n)_{n\ge0}$ by
\[
  (X_{n+1},Y_{n+1})
  = 
  \begin{cases}
    (\,X_n + V_{n+1},\,Y_n\,), 
      &\text{if } U_{n+1} < \dfrac{X_n}{X_n + Y_n}, \\[1em]
    (\,X_n,\,Y_n + V_{n+1}), 
      &\text{if } U_{n+1} \ge \dfrac{X_n}{X_n + Y_n}.
  \end{cases}
\]
For all $n\ge0$, set
\[
  M_n \;=\; \frac{X_n}{X_n + Y_n}.
\]
Show that $(M_n)_{n\ge0}$ is a martingale with respect to the filtration
\[
  \mathcal{F}_n \;=\;\sigma\bigl(U_1,V_1,\dots,U_n,V_n\bigr).
\]
\emph{(Hint: first consider the case where the $V_i$ are non-random.)}
\end{exercise}

\begin{proof}
We show that $X_n$ and $Y_n$ are $\mathcal{F}_n$-measurable by induction. For $n=0$, $X_0=x$ and $Y_0=y$ so $\sigma(X_0,Y_0)=\{\emptyset,\Omega\}\subseteq\mathcal{F}_0$. Now assume that $X_k$ and $Y_k$ are $\mathcal{F}_k$-measurable. Then \[X_{k+1}=X_k+V_{k+1}\mathbf{1}_{\{U_{k+1}<\frac{X_k}{X_k+Y_k}\}}.\] $X_k$ is $\mathcal{F}_{k+1}$-measurable by the inductive hypothesis, $V_{k+1}$ is $\mathcal{F}_{k+1}$-measurable by definition, and $\mathbf{1}_{\{U_{k+1}<\frac{X_k}{X_k+Y_k}\}}$ is $\mathcal{F}_{k+1}$ measurable since $\{U_{k+1}<\frac{X_k}{X_k+Y_k}\}\in\mathcal{F}_{k+1}$. Hence $X_{k+1}$ is $\mathcal{F}_{k+1}$ measurable. Similarly $Y_{k+1}$ is $\mathcal{F}_{k+1}$-measurable. Hence $X_n$ and $Y_n$ are both $\mathcal{F}_n$-measurable for all $n$ by induction. hence, $M_n$ is $\mathcal{F}_{n}$-measurable $\forall n$.

We now show that $M_n$ is integrable $\forall n$. Clearly $M_0$ is integrable. Otherwise,\[M_{n+1}=\frac{X_n+V_{n+1}\mathbf{1}_{\{U_{n+1}<\frac{X_n}{X_n+Y_n}\}}}{X_n+Y_n+V_{n+1}}<1.\] Hence $M_n$ is integrable $\forall n$.

\begin{align*}\mathbb{E}[M_{n+1}|\mathcal{F}_n]&=\mathbb{E}\left[\frac{X_n+V_{n+1}\mathbf{1}_{\{U_{n+1}<\frac{X_n}{X_n+Y_n}\}}}{X_n+Y_n+V_{n+1}}|\mathcal{F}_n\right]\\&=\mathbb{E}\left[\sum_{i=0}^\infty\mathbf{1}_{\{V_{n+1}=i\}}\frac{X_n+i\mathbf{1}_{\{U_{n+1}<\frac{X_n}{X_n+Y_n}\}}}{X_n+Y_n+i}|\mathcal{F}_n\right]\\&=\sum_{i=0}^\infty\mathbb{E}\left[\mathbf{1}_{\{V_{n+1}=i\}}\frac{X_n+i\mathbf{1}_{\{U_{n+1}<\frac{X_n}{X_n+Y_n}\}}}{X_n+Y_n+i}|\mathcal{F}_n\right]\\&=\sum_{i=0}^\infty\frac{\mathbb{E}[\mathbf{1}_{\{V_{n+1}=i\}}(X_n+i\mathbf{1}_{\{U_{n+1}<\frac{X_n}{X_n+Y_n}\}})|\mathcal{F}_n]}{X_n+Y_n+i}\\&=\sum_{i=0}^\infty\frac{X_n\mathbb{E}[\mathbf{1}_{\{V_{n+1}=i\}}|\mathcal{F}_n]+\mathbb{E}[i\mathbf{1}_{\{V_{n+1}=i\}}\mathbf{1}_{\{U_{n+1}<\frac{X_n}{X_n+Y_n}\}}|\mathcal{F}_n]}{X_n+Y_n+i}\\&=\sum_{i=0}^\infty\frac{X_n\mathbb{P}(V_{n+1}=i)+i\mathbb{E}[\mathbf{1}_{\{V_{n+1}=i\}}\mathbf{1}_{\{U_{n+1}<\frac{X_n}{X_n+Y_n}\}}|\mathcal{F}_n]}{X_n+Y_n+i}\\&=\sum_{i=0}^\infty\frac{X_n\mathbb{P}(V_{n+1}=i)+i\mathbb{P}(V_{n+1}=i,U_{n+1}<\frac{X_n}{X_n+Y_n}|\mathcal{F}_n)}{X_n+Y_n+i}\\&=\sum_{i=0}^\infty\frac{X_n\mathbb{P}(V_{n+1}=i)+i\mathbb{P}(V_{n+1}=i)M_n}{X_n+Y_n+i}\\&=\sum_{i=0}^\infty\mathbb{P}(V_{n+1}=i)\frac{X_n+iM_n}{X_n+Y_n+i}\\&=\sum_{i=0}^\infty\mathbb{P}(V_{n+1}=i)\frac{X_n+\frac{iX_n}{X_n+Y_n}}{X_n+Y_n+i}\\&=\sum_{i=0}^\infty\mathbb{P}(V_{n+1}=i)\frac{\frac{X_n(X_n+Y_n+i)}{X_n+Y_n}}{X_n+Y_n+i}\\&=\sum_{i=0}^\infty\mathbb{P}(V_{n+1}=i)M_n\\&=M_n\text{ almost surely.}\end{align*}
Hence $M_n$ is a martingale with respect to $(\mathcal{F}_n)_{n\geq0}$.
\end{proof}

\begin{exercise}
\begin{enumerate}
    \item[(a)] Assume that $(X_i)_{i \geq 1}$ is a sequence of random variables and 
    $\mathcal{F}_n = \sigma(X_1, X_2, \ldots, X_n)$ for all $n \geq 1$. 
    Which of the following random variables are stopping times 
    (use the convention $\min \emptyset = +\infty$)? Justify your answers.
    
    \begin{enumerate}
        \item[i.] $T_1 = \min\{n \geq 1 : X_n \in [15, 20]\}$
        \item[ii.] $T_2 = \min\{n \geq 1 : X_1 + 2X_2 + \cdots + nX_n \geq 120\}$
        \item[iii.] $T_3 = 5$
        \item[iv.] $T_4 = \min\{n \geq 2 : X_n = 8 \text{ and } X_{n-1} = 6\}$
        \item[v.] $T_5 = \min\{n \geq 1 : X_n = 8 \text{ and } X_{n+1} = 6\}$
    \end{enumerate}
    
    \item[(b)] Suppose $S$ and $T$ are stopping times. Show that $\min(S, T)$ and $S + T$ are stopping times.
\end{enumerate}
\end{exercise}
\begin{proof}
\begin{enumerate}
\item[(a)] 
\begin{itemize}
\item[(i)] \[\{T_1\leq n\}=\bigcup_{i=1}^n\{X_i\in[15,20]\}\in\mathcal {F}_n\] so $T_1$ is a stopping time.
\item[(ii)] \[\{T_2\leq n\}=\bigcup_{i=1}^n\left\{\sum_{j=1}^ijX_j\geq120\right\}\in\mathcal{F}_n\] so $T_2$ is a stopping time.
\item[(iii)] \[\{T_3=n\}\in\{\emptyset,\Omega\}\] so $T_3$ is trivially a stopping time.
\item[(iv)] \[\{T_4=1\}=\emptyset\in\mathcal{F}_1.\]And for $n\geq 2,$\[\{T_4\leq n\}=\bigcup_{i=2}^n\{X_i=8,X_{i-1}=6\}\in\mathcal{F}_n.\] Hence $X_4$ is a stopping time.
\item[(v)] \[\{T_5=1\}=\{X_1=8,X_2=6\}\not\in\mathcal{F}_1\] so $T_5$ is not a stopping time.
\end{itemize}
\item[(b)] \[\{\min(S,T)\leq n\}=\{S\leq n\}\cup\{T\leq n\}\in\mathcal{F}_n\] so $\min(S,T)$ is a stopping time.\[\{S+T=n\}=\bigcup_{i=1}^{n-1}\{S=i,T=n-i\}\in\mathcal{F}_n.\] Hence $S+T$ is a stopping time.
\end{enumerate}
\end{proof}

\begin{exercise}
An investor buys shares in a company. At the end of each six-month period, the shareholder will receive dividends of size $0$, $1$, or $2$ per share, with probability $1/6$, $1/2$, and $1/3$, respectively. Let $(X_i)_{i \geq 1}$ be the successive dividends (per share) received by the investor (assumed i.i.d.).

\medskip

The investor sells his shares at the first time when the dividend is of size $0$ (say, on the $T$-th dividend: $X_T = 0$). Let $S = \sum_{i=1}^{T} X_i$ be the total dividend income per share received by the investor before they sell. Calculate $\mathbb{E}[S]$.
\end{exercise}

\begin{exercise}
Let $(S_n)_{n \geq 0}$ be the simple random walk with parameter $p \neq 1/2$. Let $a > 0$ and $b > 0$ be integers. Let 
\[
T = \min\{n \geq 1 : S_n = a \text{ or } S_n = -b\}.
\]
Find a formula for $\mathbb{E}[T]$.

\medskip

\textit{You can use without a proof that $\mathbb{E}[T] < +\infty$. $\mathbb{P}(S_T = a) = \dfrac{\alpha^b - 1}{\alpha^{a+b} - 1}$, where $\alpha = \dfrac{1 - p}{p}$.}

\textit{(See Equation (8.5) in the lecture notes for the second claim. The first claim can be proved by adapting the proof of $\mathbb{E}[T] < +\infty$ in the symmetric case in Section 7.2.)}

\end{exercise}

\end{document}
